\documentclass[sigconf, natbib=true]{acmart}
\usepackage{float}
\usepackage{booktabs} % For formal tables
\usepackage{url}
\usepackage{color}
\usepackage{enumitem}
\usepackage{multirow}
\hyphenation{Media-Eval}

% DOI
\acmDOI{}

% ISBN
\acmISBN{}

\settopmatter{printacmref=false}

\acmConference[DE'23-'24]{Data Engineering}{Radboud University}{Nijmegen, Netherlands} 
\acmYear{2024}
\copyrightyear{2024}

\setcopyright{none}

\begin{document}
\title{Data Engineering --- Intermediate Report 2}

\author{Daan Brugmans (s1080742)}
\affiliation{%
    \institution{Radboud University} 
    \city{Nijmegen}
    \country{Netherlands}
}
\email{daan.brugmans@ru.nl}

\maketitle

\section{Data \& Project Proposal}
For the Data Engineering project, I would like to build a pipeline that serves data to data scientists working on a beer recommendation system.
The pipeline should collect from multiple sources that serve data(sets) of beer and their attributes.
Examples of attributes that should be served to the data scientists are name, brewery, alcohol by volume (ABV), international bitterness unit (IBU), category/style, textual description, flavor profile, and personal ratings.
The pipeline should serve two sets of data to the data scientists: a big general dataset containing beer data from varying publicly available sources that can (and should) be updated, and a small dataset of an end user of the beer recommendation system that includes personal ratings of beers the end user has had before.
These datasets will be served tabular.
The data scientists can then use the ratings of the end user and the data of both datasets to build a system that can recommend beers from the big dataset to the end user's non-specified preferences, which can be extracted from the end user's data using machine learning.
A realistic use case for such a recommendation system could exist for social media platforms revolving around beer and online retailers of beer that can use the system to increase revenue and beer sales.

For the small dataset of an end user, I will provide my own data of beer ratings that I have collected on \citeauthor{untappd}.
\citet{untappd} is a social medium where users rate beers they try and share their ratings with friends by registering their rating on the medium ("check-in").
I participate in \citeauthor{untappd} and have collected a dataset of these check-ins with ratings.
I can access an export of my check-in data per GDPR request.
This export contains information about a beer's name, category/style, brewery, check-in location, purchasing location, flavor profile experienced by the end user, rating, and timestamp.
The data will be provided in CSV format and contain almost 400 check-ins, slightly over 300 of which being unique, so while I do not expect for missing data to be a major issue, duplicates will be more prominent.

For the big dataset containing beers that can be recommended to an end user, I have found multiple data sources of beer attributes.
I want to aggregate the data from these sources into a single database of beers. 
Because my data sources serve data in varying formats, I expect that the aggregating of my data into a single collection will be a major challenge of the project.
For example, I will have to make sure that all data uses the same formats and standards, that there are no duplicates due to beers being present in multiple sources, and that there will likely be missing data due to some data sources not storing certain attributes.

I want to make use of the following data sources:
\begin{itemize}
    \item \citet{openbeerdb} is an archive of beer data.
    Although it currently serves an older, static dataset, \citeauthor{openbeerdb} is planning on updating this dataset in the future, possibly including regular updates to the data.
    The \citeauthor{openbeerdb} data is served as a set of \texttt{.sql} files that will create tables for beer, breweries, categories, and styles (finer granularity categories), and insert them with data.
    The resulting SQL database contains most attributes I would want to serve to the data scientists, with the exception of flavor profile.
    \item \citet{philipperemy} serves a static collection of beer data scraped from \citet{brewerydb}.
    The dataset contains over 30,000 beers stored as JSON files, where every JSON file is a collection of beers.
    The records contain almost all the data that I need, plus a lot more that I do not need, and seem quite complete.
    The dataset size is ~80MBs.
    \item \citet{evanhallmark} provides a dataset of beers, breweries, and beer reviews on Kaggle.
    However, from Kaggle's preview, I have already noticed it contains several data quality issues: no documentation, missing data, strings that represent missing data, and the file for the reviews is over 2GBs.
    The dataset consists of three CSV files, and the CSV for beer records contains almost 300,000 unique names.
\end{itemize}

Although I have access to more data sources, I expect that this proposal fits nicely into the scope of the Data Engineering project.
While only \citeauthor{evanhallmark}'s dataset shows major data quality issues on the surface, I expect that I will encounter more issues when aggregating data sources.

\section{Data Quality}
Before wrangling the data, I will talk about the data quality of the various data sources.
% Zeg of je een duidelijke primary key hebt gevonden, en of het duidelijk is welke kolommen tussen datasets semantisch hetzelfde zijn

\subsection{Untappd}
The Untappd data contains all of the features I need.
Most of these features do not have missing data: beer name, brewery name, beer type, ABV, IBU, and personal rating have 0\% data missing.
The flavor profiles experienced by the user have 1\% missing data, but since that's so little, I expect to be able to handle that issue easily.
More problematic is the feature for textual descriptions: 77\% of the check-ins do not have a textual comment, and of the remaining 23\%, not all comments are semantically relevant to the beer itself.
I expect that this feature is not one I can make good use of.

Based on box-and-whiskers-plot and barplot visualizations, seem to be no outliers present in the data.
Two features are units of measure, ABV and IBU, and they are formatted correctly: ABV as a percentage, and IBU as a float.
The check-in timestamp is formatted to the standard ISO format, and floats use periods as the decimal mark.
The dataset is in an unnormalized form.

Although the Untappd data provides a unique ID for each record, for our purposes, that is not the primary key of our data: that would be beer name + brewery name.
We will assume that a brewery does not make multiple different beers that share the same name.
Given this primary key, the Untappd dataset consists of 18\% duplicates.
This can be fixed during data wrangling: duplicates are different check-ins of the same beer.
Features like beer name and brewery will be consistent across duplicates.
For some features, we must apply some transformation. 
For example, personal ratings can be averaged, and flavor profiles of all check-ins can be merged into one list.

\subsection{OpenBeerDB}

\section{Data Wrangling}

\subsection{Untappd}
% Wrangle the beer type into two features: general beer category and specific beer style
% You can wrangle the flavor profile lists into features: every flavor in the list becomes a boolean feature. 
%   If a beer has that flavor in its flavor profile, the value is true, otherwise false.

\bibliographystyle{ACM-Reference-Format}
\def\bibfont{\small} % comment this line for a smaller fontsize
\bibliography{refs.bib} 

\end{document}
