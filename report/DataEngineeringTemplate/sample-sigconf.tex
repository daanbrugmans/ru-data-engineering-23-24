\documentclass[sigconf]{acmart}

\setcopyright{none}
\settopmatter{printacmref=false}
\renewcommand\footnotetextcopyrightpermission[1]{} 

\usepackage{listings}
\usepackage{xcolor}
\usepackage{lipsum}

\definecolor{codegreen}{rgb}{0,0.6,0}
\definecolor{codegray}{rgb}{0.5,0.5,0.5}
\definecolor{codepurple}{rgb}{0.58,0,0.82}
\definecolor{backcolour}{rgb}{0.95,0.95,0.92}

\lstset{
    backgroundcolor=\color{backcolour},   
    commentstyle=\color{codegreen},
    keywordstyle=\color{magenta},
    numberstyle=\tiny\color{codegray},
    stringstyle=\color{codepurple},
    basicstyle=\ttfamily\footnotesize,
    breakatwhitespace=false,         
    breaklines=true,                 
    captionpos=b,                    
    keepspaces=true,                 
    numbers=left,                    
    numbersep=5pt,                  
    showspaces=false,                
    showstringspaces=false,
    showtabs=false,                  
    tabsize=2
}

\acmConference[Data Engineering]{}{June 2024}{Radboud University, Nijmegen}
\acmDOI{}
\acmISBN{}
\acmYear{}
\acmSubmissionID{}

%%
%% end of the preamble, start of the body of the document source.
\begin{document}

%%
%% The "title" command has an optional parameter,
%% allowing the author to define a "short title" to be used in page headers.
\title[]{Title}

%%
%% The "author" command and its associated commands are used to define
%% the authors and their affiliations.
\author{Firstname Lastname}

%%
%% By default, the full list of authors will be used in the page
%% headers. Often, this list is too long, and will overlap
%% other information printed in the page headers. This command allows
%% the author to define a more concise list
%% of authors' names for this purpose.
\renewcommand{\shortauthors}{Firstname Lastname}

%%
%% The abstract is a short summary of the work to be presented in the
%% article.
\begin{abstract}
  \lipsum[0-1]
\end{abstract}

%%
%% This command processes the author and affiliation and title
%% information and builds the first part of the formatted document.
\maketitle

\section{Introduction}
\lipsum[1-2]

\section{Quality Assessment}
\lipsum[2-3]

\section{Reshaping/wrangling}
\lipsum[3-4]

\section{Serving}
\lipsum[4-5]

\section{Lifecycle}
\lipsum[5-6]

\section{Conclusion}
\lipsum[6-8]

\bibliographystyle{ACM-Reference-Format}
\bibliography{sample-base}

\appendix

\section{Quality Assessment}

You can use the appendix to include longer code snippets~\cite{raasveldt:2019} that do not fit or belong in the main article text. Use code listings, like the ones provided below.

\begin{lstlisting}[language=Python]
import duckdb
# your code here
\end{lstlisting}

Or for another language, like SQL:

\begin{lstlisting}[language=SQL]
SELECT COUNT(*)
WHERE col1 IS NULL
\end{lstlisting}

\end{document}
